\usepackage[utf8]{inputenc}
\usepackage{fancybox}

\usepackage{tcolorbox}
\newtcbox{\mybox}{nobeforeafter,colframe=red,colback=white,boxrule=0.5pt,arc=16pt,
    boxsep=0pt,left=6pt,right=6pt,top=6pt,bottom=6pt,tcbox raise base}

\usepackage[
  left=20mm,
  right=20mm,
  top=20mm,
  bottom=20mm,
  a4paper, % A0: 840mm x 1188mm -- A1: 594mm x 840mm -- A2: 420mm x 594mm 
           %                    -- A3: 297mm x 420mm -- A4: 210mm x 297mm
  landscape,margin=1cm]{geometry}

\usepackage{amsfonts}

%\newcommand{\generate}[2]{\genCode{#1}{#2}}
\newcommand{\generate}[2]{\genText{#1}{#2}}
\newcommand{\genCode}[2]{#1}
\newcommand{\genText}[2]{#2}
\newcommand{\ignore}[1]{}
\newcommand{\termApp}[2]{\ensuremath{\stackrel{\scriptstyle #2}{\mybox{$#1$}}}}
%\newcommand{\term}[2]{\ensuremath{\stackrel{\scriptstyle #2}{\Ovalbox{$#1$}}}}
\newcommand{\term}[2]{\ensuremath{\Ovalbox{$#1$}}}
\newcommand{\ctn}[1]{\textbf{\textrm{#1}}}

\generate{ % Typed version, generation of Haskell code
\newcommand{\chvar}[2]{\texttt{T.chvar }{\Hstring{#1}}\ \paren{#2}}
\newcommand{\chlam}[4]{\texttt{T.chlam }{\Hstring{#1}}\ \paren{#2}\ \paren{#3}\ \paren{#4}}
\newcommand{\chlamLN}[4]{\texttt{T.chlam }{\Hstring{#1}}\ \paren{#2}\ \paren{#3}\ \paren{#4}}
\newcommand{\chmuRec}[4]{\texttt{T.chmu }{\Hstring{#1}}\ \paren{#2}\ \paren{#3}\ \paren{#4}}
\newcommand{\chmuRecLN}[4]{\texttt{T.chmu }{\Hstring{#1}}\ \paren{#2}\ \paren{#3}\ \paren{#4}}
\newcommand{\chapp}[3]{\texttt{T.chapp }\paren{#1}\ \paren{#2}\ \paren{#3}}
\newcommand{\chappLN}[3]{\texttt{T.chapp }\paren{#1}\ \paren{#2}\ \paren{#3}}
\newcommand{\chpair}[2]{\texttt{T.chePar }\paren{#1}\ \paren{#2}}
\newcommand{\chpairLN}[2]{\texttt{T.chePar }\paren{#1}\ \paren{#2}}
\newcommand{\chprj}[3]{\texttt{T.chprj }\paren{#1}\ \paren{#2}\ \paren{#3}}
%
\newcommand{\chdetPrj}[3]{PROBLEMS}
%
\newcommand{\chnum}[1]{\texttt{T.chnum }\paren{#1}}
\newcommand{\chzero}{\texttt{T.chzero }}
\newcommand{\chsucc}[1]{\texttt{T.chsucc }\paren{#1}}
\newcommand{\chpred}[1]{\texttt{T.chpred }\paren{#1}}
\newcommand{\chIfZ}[4]{\texttt{T.chifZ }\paren{#1}\ \paren{#2}\ \paren{#3}\ \paren{#4}}
\newcommand{\chIfZLN}[4]{\texttt{T.chIfZ }\paren{#1}\ \paren{#2}\ \paren{#3}\ \paren{#4}}
\newcommand{\chIfEq}[5]{\texttt{T.chifEq }\paren{#1}\ \paren{#2}\ \paren{#3}\ \paren{#4}\ \paren{#5}}
\newcommand{\chIfEqLN}[5]{\texttt{T.chIfEq }\paren{#1}\ \paren{#2}\ \paren{#3}\ \paren{#4}\ \paren{#5}}
\newcommand{\chNewCanon}[3]{\texttt{T.chcanon }\paren{#1}\ \paren{#2}\ \paren{#3}}
\newcommand{\chNewUncanon}[3]{\texttt{T.chuncanon }\paren{#1}\ \paren{#2}\ \paren{#3}}
%
\newcommand{\Star}[1]{\texttt{T.starE }\paren{#1}}
%
\newcommand{\AliasA}[2]{\textrm{#1}^{#2}}
\newcommand{\AliasB}[3]{\textrm{#1}^{#2}\ ({#3})}
\newcommand{\AliasC}[4]{(\textrm{#1}{#2}\textrm{#3}^{#4})}

\newcommand{\Tmulti}[1]{\texttt{[}#1texttt{]}}
\newcommand{\TCanon}[2]{\texttt{IT.TCanon }{#1}\ \paren{#2}}
\newcommand{\Tbiotas}[1]{\texttt{IT.biotas }{#1}}
\newcommand{\Tiotas}[1]{\texttt{IT.iotas }{#1}}
\newcommand{\Tfun}[2]{\texttt{IT.tfun }\toList{#1}\ \paren{#2}}
\newcommand{\Tpar}[1]{\texttt{IT.tpar }\toList{#1}}
\newcommand{\Ttau}{\texttt{IT.tau}}
\newcommand{\TIota}{\texttt{IT.iota}}
\newcommand{\TNat}{\texttt{IT.tNat}}
}{ % Typed version, generation of LaTeX output
\newcommand{\binder}[2]{\ensuremath{{#1}^{\scriptstyle #2}}}

\newcommand{\chvar}[2]{\ensuremath{\term{#1}{\Tmulti{#2}}}}
\newcommand{\chlam}[4]{\ensuremath{\term{\lambda{\binder{#1}{\Tmulti{#2}}}.{#3}}{#4}}}
\newcommand{\chlamLN}[4]{\ensuremath{\term{\begin{tabular}[t]{l}
                                                  \ensuremath{\lambda{\binder{#1}{\Tmulti{#2}}}.} \\
                                                  \ \ \ \ensuremath{#3}
                                           \end{tabular}}{#4}}}
\newcommand{\chmuRec}[4]{\ensuremath{\term{\mu{\binder{#1}{\Tmulti{#2}}}.{#3}}{#4}}}
\newcommand{\chmuRecLN}[4]{\ensuremath{\term{\begin{tabular}[t]{l}
                                                    \ensuremath{\mu{\binder{#1}{\Tmulti{#2}}}.} \\
                                                    \ \ \ \ensuremath{#3}
                                             \end{tabular}}{#4}}}
\newcommand{\chapp}[3]{\ensuremath{\termApp{#1\ #2}{#3}}}
%\newcommand{\chapp}[3]{\ensuremath{\term{#1\ #2}{#3}}}
\newcommand{\chappLN}[3]{\ensuremath{\term{\begin{tabular}[t]{l}
                                                  \ensuremath{\paren{#1}} \\
                                                  \ \$\ \ensuremath{\paren{#2}}
                                           \end{tabular}}{#3}}}
\newcommand{\chpair}[2]{\ensuremath{\term{#1}{#2}}}
\newcommand{\chpairLN}[2]{\ensuremath{\term{\begin{tabular}[t]{l}
                                                  \ensuremath{\paren{#1}}
                                           \end{tabular}}{#2}}}
\newcommand{\chprj}[3]{\ensuremath{\term{\pi_{\scriptstyle #1}{(#2)}}{#3}}}
%
\newcommand{\chdetPrj}[3]{\ensuremath{\term{\Pi_{\scriptstyle #1}{(#2)}}{#3}}}
%
\newcommand{\chnum}[1]{\ensuremath{\underline{#1}}}
\newcommand{\chzero}{\ensuremath{\chnum{0}}}
\newcommand{\chsucc}[1]{\ensuremath{\ctn{succ}\paren{#1}}}
\newcommand{\chpred}[1]{\ensuremath{\ctn{pred}\paren{#1}}}
\newcommand{\chIfZ}[4]{\ensuremath{\term{\ctn{IfZ}\ \paren{#1}\ \paren{#2}\ \paren{#3}}{#4}}}
\newcommand{\chIfZLN}[4]{\ensuremath{\term{\begin{tabular}[t]{l}
                                             \ensuremath{\ctn{IfZ}\ \paren{#1}}  \\
                                             \ \ \ then=\ensuremath{\paren{#2}}    \\
                                             \ \ \ else=\ensuremath{\paren{#3}}
                                           \end{tabular}
                                          }{#4}}}
\newcommand{\chIfEq}[5]{\ensuremath{\term{\ctn{IfEq}\ \paren{#1}\ \paren{#2}\ \paren{#3}\ \paren{#4}}{#5}}}
\newcommand{\chIfEqLN}[5]{\ensuremath{\term{\begin{tabular}[t]{l}
                                             \ensuremath{\ctn{IfEq}\ \paren{#1}\ \paren{#2}}  \\
                                             \ \ \ then=\ensuremath{\paren{#3}}    \\
                                             \ \ \ else=\ensuremath{\paren{#4}}
                                           \end{tabular}
                                          }{#5}}}
\newcommand{\chNewCanon}[3]{\enusremath{\term{\newCanon{#1}{#2}}{#3}}}
\newcommand{\chNewUncanon}[3]{\enusremath{\term{\newUncanon{#1}{#2}}{#3}}}
%
\newcommand{\Star}[1]{\ensuremath{\star}^{#1}}
%
\newcommand{\AliasA}[2]{\ensuremath{\textrm{#1}^{#2}}}
\newcommand{\AliasB}[3]{\ensuremath{(\textrm{#1}^{#2}\ ({#3}))}}
\newcommand{\AliasC}[4]{\ensuremath{(\textrm{#1}{#2}\textrm{#3}^{#4})}}

\newcommand{\Tmulti}[1]{\ensuremath{#1}}
\newcommand{\TCanon}[2]{\ensuremath{[\!\![{#2}]\!\!]^\bar{#1}}}
\newcommand{\Tbiotas}[1]{\ensuremath{\bar{#1}}}
\newcommand{\Tiotas}[1]{\ensuremath{#1}}
\newcommand{\Tfun}[2]{\ensuremath{{#1}\rightarrow{#2}}}
\newcommand{\Tpar}[1]{\ensuremath{\Tmulti{#1}}}
\newcommand{\Ttau}{\tau}
\newcommand{\TIota}{\iota}
\newcommand{\TNat}{\ctn{Nat}}
}

% Macros to generate functional code
\newcommand{\Hstring}[1]{\texttt{"}\texttt{#1}\texttt{"}}
\newcommand{\paren}[1]{\texttt{(}\texttt{#1}\texttt{)}}
\newcommand{\toList}[1]{\texttt{[}\texttt{#1}\texttt{]}}
%
\generate{ % Untyped version, generation of Haskell code
\newcommand{\var}[2]{\texttt{T.var }{\Hstring{#1}}\ \paren{#2}}
\newcommand{\lam}[3]{\texttt{T.lam }{\Hstring{#1}}\ \paren{#2}\ \paren{#3}}
\newcommand{\lamLN}[3]{\texttt{T.lam }{\Hstring{#1}}\ \paren{#2}\ \paren{#3}}
\newcommand{\muRec}[3]{\texttt{T.mu }{\Hstring{#1}}\ \paren{#2}\ \paren{#3}}
\newcommand{\muRecLN}[3]{\texttt{T.mu }{\Hstring{#1}}\ \paren{#2}\ \paren{#3}}
\newcommand{\app}[2]{\texttt{T.app }\paren{#1}\ \paren{#2}}
\newcommand{\appLN}[2]{\texttt{T.app }\paren{#1}\ \paren{#2}}
\newcommand{\pair}[1]{\texttt{T.ePar }\paren{#1}}
\newcommand{\pairLN}[1]{\texttt{T.ePar }\paren{#1}}
\newcommand{\prj}[2]{\texttt{T.prj }\paren{#1}\ \paren{#2}}
\newcommand{\Zero}{\texttt{T.zero }}
\newcommand{\Succ}[1]{\texttt{T.succ }\paren{#1}}
\newcommand{\Pred}[1]{\texttt{T.pred }\paren{#1}}
\newcommand{\ifZ}[3]{\texttt{T.ifZ }\paren{#1}\ \paren{#2}\ \paren{#3}}
\newcommand{\ifEq}[4]{\texttt{T.ifZ }\paren{#1}\ \paren{#2}\ \paren{#3}\ \paren{#4}}
\newcommand{\newCanon}[2]{\texttt{T.canon }\paren{#1}\ \paren{#2}}
\newcommand{\newUncanon}[2]{\texttt{T.uncanon }\paren{#1}\ \paren{#2}}

\newcommand{\tfun}[2]{\texttt{IT.tfun }\toList{#1}\ \paren{#2}}
\newcommand{\tpar}[1]{\texttt{IT.tpar }\toList{#1}}
\newcommand{\tcanon}[2]{\texttt{IT.TCanon }{#1}\ \paren{#2}}
\newcommand{\tbiotas}[1]{\texttt{IT.biotas }{#1}}
\newcommand{\tiotas}[1]{\texttt{IT.iotas }{#1}}
\newcommand{\ttau}{\texttt{IT.tau}}
\newcommand{\tIota}{\texttt{IT.iota}}
\newcommand{\tNat}{\texttt{IT.tNat}}
}{ % Untyped version, generation of LaTeX output
\newcommand{\var}[2]{\ensuremath{{#1}^{\scriptstyle #2}}}
\newcommand{\lam}[3]{\ensuremath{\lambda{\var{#1}{#2}}.{#3}}}
\newcommand{\lamLN}[3]{\begin{tabular}[t]{@{}l@{}}
                        \ensuremath{\lambda{\var{#1}{#2}}.}
                        \\\ \ \ \ \ensuremath{{#3}}
                       \end{tabular}}
\newcommand{\muRec}[3]{\ensuremath{\mu{\var{#1}{#2}}.{#3}}}
\newcommand{\muRecLN}[3]{\begin{tabular}[t]{@{}l@{}}
                        \ensuremath{\mu{\var{#1}{#2}}.}
                        \\\ \ \ \ \ensuremath{{#3}}
                       \end{tabular}}
\newcommand{\app}[2]{\ensuremath{(#1)\ (#2)}}
\newcommand{\appLN}[2]{\begin{tabular}[t]{@{}l@{}}
                         \ensuremath{(#1}\\
                         \ensuremath{)\ (#2)}
                       \end{tabular}}
\newcommand{\pair}[1]{\ensuremath{(#1)}}
\newcommand{\pairLN}[1]{\begin{tabular}[t]{@{}l@{}}
                         \ensuremath{(#1)}\\
                       \end{tabular}}
\newcommand{\prj}[2]{\ensuremath{\pi_{\scriptstyle #1}{(#2)}}}
\newcommand{\Zero}{\ensuremath{0}}
\newcommand{\Succ}[1]{\ensuremath{\ctn{succ}\ \paren{#1}}}
\newcommand{\Pred}[1]{\ensuremath{\ctn{pred}}\ \paren{#1}}
\newcommand{\ifZ}[3]{\ensuremath{\ctn{ifZ}\ \paren{#1}\ \paren{#2}\ \paren{#3}}}
\newcommand{\ifEq}[4]{\ensuremath{\ctn{ifZ}\ \paren{#1}\ \paren{#2}\ \paren{#3}\ \paren{#4}}}
\newcommand{\newCanon}[2]{\ensuremath{[\!\![{#2}]\!\!]^\bar{#1}}}
\newcommand{\newUncanon}[2]{\ensuremath{{#2}\Star{#1}}}

\newcommand{\tfun}[2]{\ensuremath{\Tmulti{#1}\rightarrow{#2}}}
\newcommand{\tpar}[1]{\ensuremath{\Tmulti{#1}}}
\newcommand{\tcanon}[2]{\ensuremath{[\!\![{#2}]\!\!]^\bar{#1}}}
\newcommand{\tbiotas}[1]{\ensuremath{\bar{#1}}}
\newcommand{\tiotas}[1]{\ensuremath{#1}}
\newcommand{\ttau}{\tau}
\newcommand{\tIota}{\iota}
\newcommand{\tNat}{\ctn{Nat}}
}

% QComplex this is not fine
%\newcommand{\QComplex}[4]{\ensuremath{\paren{#1 + #2 \sqrt{2} + #3 i + #4 i \sqrt{2}}}}

% QTypes
\newcommand{\BaseQMoreBeautiful}{\ensuremath{I\!\!B}}
\newcommand{\BaseQ}{\ensuremath{\mathbb B}}
\newcommand{\BaseQN}[1]{\ensuremath{\BaseQ^{#1}}}
\newcommand{\TSup}[1]{\ensuremath{S(#1)}}
\newcommand{\TFun}[2]{\ensuremath{#1 \Rightarrow #2}}
\newcommand{\TProd}[2]{\ensuremath{#1 \times #2 }}


% QTerms
\newcommand{\ket}[1]{\ensuremath{|#1\rangle}}
\newcommand{\ketZero}[1]{\ket{0}}
\newcommand{\ketOne}[1]{\ket{1}}
\newcommand{\Null}{\ensuremath{0}}
\newcommand{\LinQT}[2]{\ensuremath{#1\ #2}}

\newcommand{\reduction}[1]{\ensuremath{\rightarrow_{#1}}}


